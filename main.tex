\documentclass[12pt, a4paper]{exam}

% Use the geometry package to set the page margins
\usepackage[margin=1in]{geometry}

% Define the rand counter
\newcounter{rand}

\begin{document}

\vspace{7mm}
% \makebox[\textwidth]{Name:\enspace\hrulefill}
% \vspace{1cm}
% \makebox[\textwidth]{Teacher:\enspace\hrulefill}
%\thispagestyle{empty}
	\noindent
	\begin{minipage}[l]{.25\textwidth}
		\noindent
		\includegraphics[width=1\textwidth]{images/logo.jpg}
	{\large \begin{center}
\Large	
General Factoring


	\end{center} 
	}
	\end{minipage}
\hfill
\begin{minipage}[c]{0.75\textwidth} \large
  \hspace{.3cm}  Students Name :\hspace{6cm} Percentage:
	\begin{center}
\addpoints
\hpword{Marks}
\gradetable[h][questions]
\noindent
	\end{center}
\end{minipage}
\vspace{0.14in}
\begin{questions}
\pointsinrightmargin
\large
\question[6]
\textbf{Product of a pair of  positive factors. Expand }
\vspace{2.4cm}
\begin{parts}
\part      (5x+7)(11x+2)      \vspace{3cm} \n
\part      (7x+5)(11x+2)      \vspace{3cm} \n
\part      (5x+2)(7x+11)      \vspace{3cm} \n
\part      (2x+7)(5x+3)       \vspace{3cm} \n
\part      (3x+11)(2x+5)      \vspace{3cm} \n
\part      (7x+5)(11x+3)      \vspace{3cm} \n
\part      (2x+5)(3x+7)       \vspace{3cm} \n
\end{parts}



\newpage

\Large
\question[6]

\Large

\textbf{Product of a pair of  positive factors. Expand }
\vspace{2.4cm}
\begin{parts}
\part 6x^2 + 47x + 91 = 0   
\vspace{2.4cm}

\part 6x^2 + 43x + 55 = 0     \vspace{2.4cm}

\part 4x^2 + 24x + 35 = 0     \vspace{2.4cm}

\part 9x^2 + 54x + 65 = 0     \vspace{2.4cm}

\part 4x^2 + 24x + 35 = 0     \vspace{2.4cm}

\part 4x^2 + 26x + 22 = 0     \vspace{2.4cm} 

\end{parts}











\end{questions}

\end{document}

