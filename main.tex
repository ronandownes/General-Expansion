\documentclass[12pt, a4paper]{exam}
\usepackage{amsmath}
\usepackage{amssymb}
\usepackage{mathtools}
\usepackage{breqn}
\usepackage{siunitx}
\usepackage{physics}
\usepackage{tikz-cd}

\usepackage{graphicx}
% Use the geometry package to set the page margins
\usepackage[margin=1in]{geometry}

% Define the rand counter
\newcounter{rand}

\begin{document}

\vspace{7mm}
% \makebox[\textwidth]{Name:\enspace\hrulefill}
% \vspace{1cm}
% \makebox[\textwidth]{Teacher:\enspace\hrulefill}
%\thispagestyle{empty}
	\noindent
	\begin{minipage}[l]{.25\textwidth}
		\noindent
		\includegraphics[width=1\textwidth]{logo.png}
	{\large \begin{center}
\Large	
General Factoring


	\end{center} 
	}
	\end{minipage}
\hfill
\begin{minipage}[c]{0.75\textwidth} \large
  \hspace{.3cm}  Students Name :\hspace{6cm} Percentage:
	\begin{center}
\addpoints
\hpword{Marks}
\gradetable[h][questions]
\noindent
	\end{center}
\end{minipage}
\vspace{0.14in}




\begin{centering}
\begin{questions}




\pointsinrightmargin
\large
\question[6]
\textbf{Product of a pair of  positive factors. Expand }

\begin{parts}
\part      (5x+7)(11x+2)      \vspace{22mm} \n
\part      (7x+5)(11x+2)      \vspace{22mm} \n
\part      (5x+2)(7x+11)      \vspace{22mm} \n
\part      (2x+7)(5x+3)       \vspace{22mm} \n
\part      (3x+11)(2x+5)      \vspace{22mm} \n
\part      (7x+5)(11x+3)      \vspace{22mm} \n
\part      (2x+5)(3x+7)       \vspace{22mm} \n
\end{parts}



\newpage

\Large
\question[6]

\Large

\textbf{Product of a pair of  positive factors. Expand }

\begin{parts}
\part      (7x-3)(2x-5)   
                    \vspace{22mm}\n
\part      (3x-2)(7x-5)   
                    \vspace{22mm}\n
\part      (5x-7)(3x-2)   
                    \vspace{22mm}\n
\part      (3x-2)(7x-5)   
                    \vspace{22mm}\n
\part      (2x-5)(3x-7)   
                    \vspace{22mm}\n
\part      (2x-3)(7x-5)   
                    \vspace{22mm}\n

\end{parts}











\end{questions}
\end{centering}

\end{document}

